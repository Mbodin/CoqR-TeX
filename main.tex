\newcommand{\citestyle}[1]{}
\documentclass[9pt, sigplan, natbib=false, screen=true]{acmart}
\let\citename\relax

\settopmatter{printacmref=false}
\renewcommand\footnotetextcopyrightpermission[1]{}
%\pagestyle{plain}
\settopmatter{printfolios=true}

\acmConference[CoqPL'18]{The Fourth International Workshop on Coq for Programming Languages}{January 2018}{Los Angeles, United States}
\acmYear{2018}
\copyrightyear{2018}

%\usepackage{balance}

%\usepackage{fontspec}

\usepackage[
   backend=biber,
   bibencoding=utf8,
   style=alphabetic,
   hyperref=true,
   % citestyle=authoryear-comp,
   backref=false,
   sortlocale=en,
   url=true,
   doi=false,
   eprint=false
 ]{biblatex}
\addbibresource{biblio.bib}

\newcommand\Coq{Coq}
\newcommand\OCaml{OCaml}
\newcommand\R{R}
\newcommand\Cn{C}

% It seems to be needed to include minted properly…
\makeatletter
\def\mdseries@tt{m}
\makeatother

\usepackage{minted}
\setminted{encoding=utf8}

\usepackage{tikz}
\usetikzlibrary{positioning}

\tikzset{
	box/.style = {
		draw = black,
        fill = white,
		rectangle,
		rounded corners = 2pt,
		text centered,
		minimum height = 5mm,
		minimum width = 10mm
	}
}

\setcopyright{none}

\title{A \Coq{} Formalisation of a Core of \R{}}
\author{Martin Bodin}
\email{mbodin@dim.uchile.cl}
\affiliation{
    \institution{Center of Mathematical Modeling}
    \streetaddress{Beauchef 851}
    \city{Santiago}
    \country{Chile}
}

\begin{document}

\begin{abstract}
Real-world programming languages have subtle behaviours.
In particular, their semantics often contains various corner cases
that programmers are unaware of,
possibly yielding to serious bugs.
This is an opportunity for our community
as \Coq{} enables to certify program behaviours.
Such certifications rely on a formal semantics,
% of the programming language of the certified program,
which can sometimes be as difficult to trust as the original program.
Previous work proposed ways to trust such large semantics.
This work evaluates the feasibility of such a formalisation
in the case of the \R{} programming language---%
a trending programming language specialised in statistics.
We introduce a formalisation of the core of \R{}
related to the reference \R{} interpreter.
\end{abstract}

\maketitle

\section{Introduction}
\label{sec:introduction}

\R{}~\parencite{R, ihaka1996r, Rwebsite} is a trending
programming language for statisticians.
It is used in a large range of fields (biology, finance, etc.),
as illustrated by the list of available packages categories:
\url{https://cran.r-project.org/web/views/}.
This diversity is reflected among \R{} programmers,
resulting in largely different programming styles.
This makes \R{} unlikely to disappear
in the following years.

The \R{} programming language presents itself
as an expressive and powerful language,
able to express complex notions in few keystrokes.
This sometimes come with the cost of readability.
% it can be difficult to fully understand what a given \R{} program exactly does.
The semantics of \R{} is subtle
and contains numerous corner cases,
as shown in Section~\ref{sec:presentation}.
%
As a consequence, bugs can be frequent in \R{} programs
(as illustrated by the existence of various debugging tools~\parencite{mcpherson2014}),
and trusting such programs can be difficult.
Formal methods in general and the \Coq{} proof assistant in particular
offer an interesting answer to this trust issue.
But to formally prove that an \R{} program meets its specification,
we need a semantics of \R{}.
% Such a semantics needs to be trustworthy.
% In particular, it should capture all the corner cases
% with which a proven program can cope.

A semantics for the full \R{} language will inevitably be complex,
as all corner cases need to be covered.
% The question of trusting such a semantics
% then naturally arises.
%
To be able to trust such a formalisation,
it is important to identify \emph{trust sources}.
These can be language specifications,
test suites,
or reference interpreters.
%
% When formalising a complex programming language such as \R{},
It can be difficult to convince non-\Coq{}-users
that they can trust our formalisation~\parencite{leroy2014pip}:
the most time-consuming part of the formalisation process is to
relate the formalisation with its trust sources,
not actually defining the formal model itself.

% Specifications are the preferred trust source.
% They provide a non-ambigous description of the language constructs's behaviours.
% A formalisation can be related by similarity:
% its constructs and operations should mimic the ones
% of the specification.
% Ideally, such a similarity would be a line-to-line closeness,
% each line of the formalisation referring to a line of the specification,
% and conversely.

% Test suites provide a large number of programs and their expected result.
% When the formalisation is executable, it can be tested against such tests suites.
% This technique then enables to evaluate the quality of the test suite
% using its code coverage.

% Reference interpreters are particularly interesting as
% they can be used similarly to either specifications or test suites.
% It is indeed possible to relate a formal semantics
% to the source code of the reference interpreter,
% using it as a specification.
% But it is also possible to compare the results of the reference interpreter
% and the formal semantics (if executable) on random programs,
% until the wanted code coverage is reached.

In the case of \R{}, there is unfortunately
no precise language specification
(there exists an ongoing effort to write one,
but it is too early to base ourself on it).
There are however test suites~\parencite{2014testr, maj2013testr},
as well as a reference interpreter~\parencite{Rwebsite}
against which alternative implementations of \R{} are tested.
%
To trust our formalisation, we wrote it in a similar
way than the source code of the reference interpreter.
We also made it executable to be able to run it on test suites.
We believe that these two relating methods are enough
to convince anyone that we closely captured all corner cases of \R{}.

The \R{} language is large,
but it contains a core which is both small and easy to identify.
We introduce a full \Coq{} specification for this core.
The formalisation presents itself as a monadic interpreter.
This interpreter mimics the operations of the reference interpreter,
using its \Cn{} source code as a specification.
%
Section~\ref{sec:presentation} presents some subtleties of the \R{} programming language.
Then Section~\ref{sec:formalisation} presents how our formalisation is defined,
and in particular, how it is related to the \R{} source code.
The source of this project is available at \url{https://github.com/Mbodin/proveR/tree/CoqPL2018Final}.


\section{Presentation of \R{} and its Core}
\label{sec:presentation}

\R{} is a weakly typed programming language.
Its basic data types mainly consist of various kinds of arrays.
It was originally~\parencite{ihaka1996r} defined as a variant of Scheme
which has been mutated in order to look like its predecessor,
the S programming language.
This construction has several implications.
First, functions are first class in \R{}.
Second, \R{} follows the code-is-data paradigm:
it is possible to delay the evaluation of statements,
and even to manipulate their inner structure
(for instance using the \mintinline{R}{substitute} \R{} function).

Almost all constructs of \R{} are treated as functions.
This includes features like \mintinline{R}{if}, \mintinline{R}{while},
and assignments.
\R{} source code contains a table,
the \emph{symbol table},
associating each construct to a \Cn{} function.
Some of these functions (\mintinline{R}{if}, assignments, etc.)
are special and directly manipulate the abstract syntax tree
of their arguments.
% The other \Cn{} functions are called after the evaluation of the arguments,
% and thus receive a basic type of the language.
%
In this work,
we consider that the features of this table
are \emph{not} part of the core of \R{}.
In other words, we define the core of \R{}
as the parts of \R{} needed to access this table
and execute its \Cn{} functions.

This definition of \R{}'s core enables us to focus
our formalisation effort on a very restricted part of the language.
It includes the execution rules for function calls,
environments, closures, promises (delayed evaluation),
as well as the parts initialising the symbol table.
Constructs like \mintinline{R}{if} and \mintinline{R}{while}
are not part of the core,
but (some kinds of) assignments are,
as they are used for function calls.
This core is easily extendable:
we can add any other feature by implementing
its associated function
and adding it to the symbol table.

We now present some corner cases of the \R{} semantics.
% A more detailed presentation can be found in~\parencite{stadler2016optimizing}.
Most \R{} constructs
% adapt themselves when given unusual arguments to,
try, in some ways, to ``guess'' what the programmer meant.
% For instance, the \mintinline{R}{i:j} construct
% builds a numerical array from the values \mintinline{R}{i}
% to \mintinline{R}{j}, which can be any expression.
% For instance \mintinline{R}{1:3} is the array containing
% the values \(1\), \(2\), and \(3\).
% But if the second argument if less than the first argument,
% \R{} builds a decreasing array instead of an empty one.
% This is a frequent bug pattern when such an expression is involved in a loop.
The same syntactic construct is thus often associated
with different behaviours.
For instance, array look-up is written \mintinline{R}{a[i]}:
if \mintinline{R}{i} evaluates to an array of indices
(numbers are considered to be arrays of size \(1\)),
then \mintinline{R}{a[i]} is the array with
the corresponding elements of \mintinline{R}{a}.
But if \mintinline{R}{i} evaluates to an array of negative integers,
then the array \mintinline{R}{a} is copied,
and all elements whose (opposite) index is in the array \mintinline{R}{i}
are removed.
% For instance, \mintinline{R}{a[-3]} is the same as the array \mintinline{R}{a},
% but without its third element
% (the later elements being moved by one index).
Other behaviours happen when \mintinline{R}{i}
evaluates to a boolean array or a string array.
As \R{} is untyped,
a function like \mintinline{R}{function (a, i) a[i]}
is already complex to specify.

Another source of subtleties of \R{} is that it lazily evaluates expressions,
including those with side effects.
For instance, the following function \mintinline{R}{f}
only evaluates its second argument if the first is \(1\).
Thus, in its second call, the variable \mintinline{R}{b}
has confusingly not been defined because the assignment \mintinline{R}{b <- 1}
was not evaluated.
\begin{minted}{R}
f <- function (x, y) if (x == 1) y
f (1, a <- 1); a # Returns 1.
f (0, b <- 1); b # Raises an error.
\end{minted}

\R{} contains numerous semantic exceptions. %~\parencite{stadler2016optimizing}.
None are inherently complex to deal with,
but their quantity complexifies the language.


\section{Formalisation}
\label{sec:formalisation}

Our formal semantics is related to the \Cn{} source code
of the \R{} reference interpreter
by a line-to-line correspondence:
every one or two lines of our semantics
correspond to one or two lines of the reference interpreter.
Our formal semantics is presented as a monadic interpreter in \Coq{}.
This interpreter is runnable,
which shall unable us to test our semantics against
\R{} test suites in future work.
However, as the core of \R{} is very limited,
non-core features will have to be added for testing.
These features can be added following the same methodology than the one presented here.
Figure~\ref{fig:coq} shows our \Coq{} translation
of the \Cn{} function shown in Figure~\ref{fig:c}.

We used monads very similar to the ones of JSRef~\parencite{popl14jscert}:
a \mintinline{Coq}{result} is either a success---%
it then transports the actual result and the resulting state
\mintinline{Coq}{S} of the program---,
an error, or \mintinline{Coq}{result_bottom}---%
meaning that the interpreter ran out of fuel during the execution.
These monads enable us to abstract most of the aspects
of imperativity and memory handling from \Cn{}.
Examples of monads include
\mintinline{Coq}|let%success|,
to extract the result given by a function
(it exactly corresponds to the monadic \emph{bind} operator),
or \mintinline{Coq}|read%defined|, to read the content of a pointer.
\Coq{} notations have heavily been used
for the line-to-line correspondence.
For instance, let us consider
the line \mintinline{C}{if (rho->type != ENVSXP)}
of Figure~\ref{fig:c}.
It has been translated into two lines:
the monad \mintinline{Coq}|read%defined|
first dereferences the pointer \mintinline{C}{rho}
(which may fail if the pointer is unbound),
then the test is performed.
Every line has been similarly translated.

\begin{figure}
\begin{minted}{C}
SEXP applyClosure (SEXP op, SEXP arglist, SEXP rho){
  SEXP formals, actuals, savedrho, newrho, res;
  if (rho->type != ENVSXP)
    error ("'rho' must be an environment.");
  formals = op->clo.formals;
  savedrho = op->clo.env;
  PROTECT (actuals = matchArgs (formals, arglist));
  /* ... */
  return res;
}
\end{minted}
    \vspace{-2mm} % There are too much blank in the last line of the code, leading
                  % to too much space.
    \caption{The original \Cn{} function}
    \label{fig:c}
\end{figure}

We chose to ignore some aspects of the original \Cn{} source code
during the formalisation.
For instance, the support for various locales and character encodings:
our interpreter only considers ASCII strings.
More importantly, the garbage collector of the reference interpreter
has not been translated.
We indeed consider that, as it is assumed not to change the final result,
its formalisation is not needed to prove the functional correctness
of \R{} programs.
For instance,
the \mintinline{C}{PROTECT} macro from Figure~\ref{fig:c}
temporarily disables the garbage collector:
it has thus been formalised out in Figure~\ref{fig:coq}.
% We believe theses change not to affect the usability of the formalisation.

Instead of defining such an interpreter,
we could have used already existing \Coq{} semantics
for \Cn{}~\parencite{formalin, Leroy-Compcert-CACM}
to directly define a semantics for \R{}.
But because of the abstraction layers of \Cn{},
proving properties about a given \R{} program
would then have been quite complex.
Furthermore, the \R{} reference interpreter has been built
with a strong intuition in mind~\parencite{ihaka1996r}.
This intuition is reflected in \R{} internals,
which would have been hidden if we attempted such a direct approach.
%
As a future work, it would be a very valuable contribution
to use these semantics to relate
our formal model to the source code of \R{}.
We furthermore believe that our line-to-line correspondence
would greatly help such a proof.

\begin{figure}
\begin{minted}{Coq}
Definition applyClosure S (op arglist rho : SExpRec_pointer)
    : result SExpRec_pointer :=
  read%defined rho_ := rho using S in
  ifb type rho_ <> EnvSxp then
    result_error S "'rho' must be an environment."
  else
    read%clo op_clo := op using S in
    let formals := clo_formals op_clo in
    let savedrho := clo_env op_clo in
    let%success actuals :=
      matchArgs S formals arglist using S in
    (* ... *)
    result_success S res.
\end{minted}
    \caption{The \Coq{} translation}
    \label{fig:coq}
\end{figure}


\section{Conclusion and Future Work}
\label{sec:conclusion}

In this work, we presented a \Coq{} monadic interpreter
mimicking the behaviours of the \R{} reference interpreter.
We believe that our approach provides a high level of trust
for our formal semantics,
on which new projects about \R{} can be based---%
for instance
proving that a given \R{} program meets its specification.
The usability of this work can be improved:
to this end, we plan to make another formalisation
of \R{} proven correct with respect to the current one.
In essence, this new formalisation should
remove all the concepts of the current formalisation coming from \Cn{}:
the semantics of \R{} being mostly functional,
most pointers can be removed in our semantics
to make it more adapted to the \Coq{} paradigms.
% We also plan to test our formalisation.
% Adding non-core features is also planned.

% \section{Related Work}
% \label{sec:related:work}
%
% There already have been a large amount of \Coq{} formalisations
% of real-world languages.
%
% Formalin.
% CompCert.
%
% JSCert.
%
% K and its \Coq{} extraction.
% K-Java.

% Say that this approach may be scalable to other projects.
% Languages with reference interpreters:
% PHP, Python, etc.

\printbibliography

\end{document}

