\newcommand{\citestyle}[1]{}
\documentclass[9pt, sigplan, natbib=false]{acmart}
\let\citename\relax

%\settopmatter{printacmref=false}
\renewcommand\footnotetextcopyrightpermission[1]{}
%\pagestyle{plain}

\acmConference[CoqPL'18]{The Fourth International Workshop on Coq for Programming Languages}{January 2018}{Los Angeles, United States}
\acmYear{2018}
\copyrightyear{2018}

\usepackage{balance}

%\usepackage{fontspec}

\usepackage[
   backend=biber,
   bibencoding=utf8,
   style=alphabetic,
   hyperref=true,
   % citestyle=authoryear-comp,
   backref=false,
   sortlocale=en,
   url=true,
   doi=false,
   eprint=false
 ]{biblatex}
\addbibresource{biblio.bib}

\newcommand\Coq{Coq}
\newcommand\OCaml{OCaml}
\newcommand\R{R}
\newcommand\Cn{C}

\usepackage{minted}
\setminted{encoding=utf8}

\usepackage{tikz}
\usetikzlibrary{positioning}

\tikzset{
	box/.style = {
		draw = black,
        fill = white,
		rectangle,
		rounded corners = 2pt,
		text centered,
		minimum height = 5mm,
		minimum width = 10mm
	}
}

\setcopyright{none}

\title{Building and Trusting \R{} Using \Coq{}}
\author{Martin Bodin}
\email{mbodin@dim.uchile.cl}
\affiliation{
    \institution{Center of Mathematical Modeling}
    \streetaddress{Beauchef 851}
    \city{Santiago}
    \country{Chile}
}

\begin{document}

\begin{abstract}
Real-world programming languages have subtil behaviours.
In particular, their semantics is often associated with
various exceptions.
Not all programmers are aware of these exceptions,
which can yield to serious bugs.
This is an opportunity for our community
as \Coq{} provides a way to certify the behaviour of a program.
But if the certified program is not written in \Coq{},
such a certification relies on a formal semantics of its programming language.
In the case of a real-world language, such a semantics
can be as difficult to trust as the original program.
Some previous work proposed ways to trust such large semantics.
This work evaluates the feasibility of such a large-scale formalisation
in the case of the \R{} programming language—%
a trending programming language specialised for statistics computations.
\end{abstract}

\maketitle

\section{Introduction}

Quick presentation of \R{}
Very diverse usage: from Big Data to finance or biology.
See the list of \R{} views (categories of packages available for \R{}):
\url{https://cran.r-project.org/web/views/}.

\R{} subtleties.

\mintinline{R}{i:j}

\begin{minted}{R}
v[i]
\end{minted}

None of these semantic exceptions are inherently complex to deal with:
it is the large amount of such exceptions that makes the language complex.

We need a semantics.
We want to trust it (reuse abstract).

Discussion about trust sources.

\section{Related Work}

Formalin.
CompCert.

JSCert.

K and its \Coq{} extraction.
K-Java.

\section{Formalisation}

Trust source used: eyeball closeness.

Description of the closeness.

\section{Conclusion and Further Work}

The full project graph.
Show that other trust sources are possible.

Say that this approach may be scalable to other projects.

\printbibliography

\end{document}

