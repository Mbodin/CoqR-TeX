
\documentclass[
    sigplan,
    10pt,
    review, % Note: remove the [review] option for the final document.
    natbib=false % Note: This ishere to be able to use Biber.
 ]{acmart}
\let\citename\relax

\settopmatter{printfolios=true,printccs=false,printacmref=false}

\acmConference[DLS'18]{Dynamic Languages Symposium}{November~6, 2018}{Boston, MA, USA}
\acmYear{2018}
\acmISBN{} % \acmISBN{978-x-xxxx-xxxx-x/YY/MM}
\acmDOI{} % \acmDOI{10.1145/nnnnnnn.nnnnnnn}
\startPage{1}

\setcopyright{none}

\bibliographystyle{ACM-Reference-Format}

\usepackage{booktabs}
\usepackage{subcaption}

\usepackage[
   backend=biber,
   bibencoding=utf8,
   style=alphabetic,
   hyperref=true,
   % citestyle=authoryear-comp,
   backref=false,
   sortlocale=en,
   url=true,
   doi=false,
   eprint=false
 ]{biblatex}
\addbibresource{biblio.bib}

\usepackage{minted}
\setminted{encoding=utf8}

\usepackage{tikz}
\usetikzlibrary{positioning}

\tikzset{
	box/.style = {
		draw = black,
        fill = white,
		rectangle,
		rounded corners = 2pt,
		text centered,
		minimum height = 5mm,
		minimum width = 10mm
	}
}

\usepackage{todonotes}
\newcommand{\mb}[1]{\todo[color=purple!20,size=\scriptsize]{#1}}
\newcommand{\mbi}[1]{\todo[color=purple!20,inline]{#1}}

\begin{document}

\title{Relating an R Formalization to R} % I didn't thought much about this. Any suggestion?

\author{Martin Bodin}
\orcid{0000-0003-3588-3782}
\affiliation{
  %\position{}
  %\department{Department1}
  \institution{Center of Mathematical Modeling}
  \streetaddress{Beauchef 851}
  \city{Santiago}
  % \state{State1}
  % \postcode{Post-Code1}
  \country{Chile}
}
\email{mbodin@dim.uchile.cl}

\author{Tomás Diaz}
% \authornote{with author2 note}
% \orcid{nnnn-nnnn-nnnn-nnnn}
\affiliation{
  % \position{Position2a}
  \department{DCC}
  \institution{Universidad de Chile}
  \streetaddress{Beauchef 851}
  \city{Santiago}
  % \state{State2a}
  % \postcode{Post-Code2a}
  \country{Chile}
}
\email{tdiaz@dcc.uchile.cl}

\author{Éric Tanter}
% \authornote{with author2 note}
% \orcid{nnnn-nnnn-nnnn-nnnn}
\affiliation{
  % \position{Position2a}
  \department{DCC}
  \institution{Universidad de Chile}
  \streetaddress{Beauchef 851}
  \city{Santiago}
  % \state{State2a}
  % \postcode{Post-Code2a}
  \country{Chile}
}
\email{etanter@dcc.uchile.cl}

\begin{abstract}
Text of abstract \ldots.
\end{abstract}

% %% 2012 ACM Computing Classification System (CSS) concepts
% %% Generate at 'http://dl.acm.org/ccs/ccs.cfm'.
\begin{CCSXML}
  <ccs2012>
    <concept>
      <concept_id>10003752.10010124.10010131.10010133</concept_id>
      <concept_desc>Theory of computation~Denotational semantics</concept_desc>
      <concept_significance>500</concept_significance>
    </concept>
    <concept>
      <concept_id>10011007.10011006.10011066.10011070</concept_id>
      <concept_desc>Software and its engineering~Application specific development environments</concept_desc>
      <concept_significance>300</concept_significance>
    </concept>
    <concept>
      <concept_id>10011007.10011074.10011099.10011692</concept_id>
      <concept_desc>Software and its engineering~Formal software verification</concept_desc>
      <concept_significance>100</concept_significance>
    </concept>
  </ccs2012>
\end{CCSXML}

\ccsdesc[500]{Theory of computation~Denotational semantics}
\ccsdesc[300]{Software and its engineering~Application specific development environments}
\ccsdesc[100]{Software and its engineering~Formal software verification}
% %% End of generated code

\keywords{R, Coq, Formalization, Testing}

\maketitle

\section{Introduction}
\label{sec:intro}

% R is used a lot.
The R programming language~\parencite{R, ihaka1996r, Rwebsite}
has gotten a lot of attention in the recent years.
It is indeed used in a large range of fields (biology, finance, etc.),
and its community ranges over millions of users.
This diversity is reflected among R programmers,
resulting in largely different programming styles.
The language itself is community driven and reflects this diversity.

% R is complex and we need to certify R softwares.
The R programming language is meant to be both expressive and powerful,
able to express complex notions in few keystrokes.
This sometimes comes with the cost of readability.
The semantics of R is subtle and contains numerous corner cases
which can be unexpected.
The reasons for these corner cases are numerous
and are often due to backward compatibility.
Another common reason is that R
is both used as a programming language and interactively:
languages expectations are usually different in these two situations.

An example of such unexpected behaviors can be found
in R function calls.
There are many ways to call a function,
but to simplify\footnote{
    We here ignore the \mintinline{R}{...} notation
    as well as default arguments.
    They have non-trivial interactions with the calling methods
    described here.
} there are two ways to provide an argument:
either by position or by name.
In the example below, the first call to \mintinline{R}{f}
associates arguments by position, and the second by name.
These two associating ways can be mixed together in the same call.
%
Association by name can be made by prefix:
the third call associates \mintinline{R}{d} to \mintinline{R}{de}
because it is the only matching argument by prefix.
If more than one argument matches by prefix,
R rejects the call and throws an error,
as in the fifth call in the example below.
However, exact matches are not counted in this process:
in the fourth call below,
the name \mintinline{R}{ab} is an exact match
and thus only the argument \mintinline{R}{abc}
is left to be associated to \mintinline{R}{a},
leading to no error thrown.
\begin{minted}{R}
# The function f concatenates its 3 arguments.
f <- function (abc, ab, de) { c (abc, ab, de) }
# The next 4 calls below returns the same vector.
f (1, 2, 3) # Association by position.
f (de = 3, abc = 1, ab = 2) # By name.
f (1, d = 3, 2) # Mixed ways.
f (3, a = 1, ab = 2) # a is associated to abc.
# The next call to f throws an error.
f (a = 3, 1, 2) # Several prefixes.
\end{minted}

Such subtle behaviors are numerous in R~\parencite{RInferno}.
Debugging tools exist~\parencite{mcpherson2014},
but they can't always compensate R's complex semantics.
Consequently, bugs occurs in R programs
and fully trusting such programs can be difficult.

% We need a formalization of the language.
Formal methods offer a solution to this trust issue:
proof assistants such as Coq~\parencite{Coq} enable us
to formally prove program properties with a high amount of trust.
But to formally prove that an R program meets its specification,
we need a formal semantics of R.
In particular, we want to catch all the subtle cases of R semantics,
such as implicit type conversions.
These corner cases are indeed a typical place were bugs appears.
Such a semantics for the full language will inevitably be complex
because of the quantity of such cases.

% This formalization is quite large, and we need to certify it.
This opens a trust problem:
how can such a large semantics (and hence the proofs made from it)
can be trusted if it is that complex?
To be able to trust such a formalization,
we need to relate to the R reference interpreters, GNU~R.
This relation between the formalism and trust sources
is not always considered to be an important part of the formalization process,
but it often needs a large amount of work~\parencite{leroy2014pip}.
Given the size of our formalization,
we consider this part to be the most important of our contributions.

% Contributions.
We introduce CoqR\mb{Are we fixed on the name?},
a formalization of the R programming language in the Coq proof assistant.
We also provide two methods to relate this formalization
to the R reference interpreter.
First, our semantics has been written in a way mimicking R source code.
Second, our semantics is executable and we have extensively tested it
against the reference interpreter.
%
This two-factors method is very close to the one of JSCert,
which we discuss in Section~\ref{sec:related:work}.

% Outlines.
This paper is organized as follows.
Section~\ref{sec:coq:interp} presents the Coq denotational semantics.
In particular, Section~\ref{sec:eyeball:closeness} presents
how this semantics is syntactically close to the C source code of R.
Section~\ref{sec:testing:architecture} then presents our testing architecture.
Not only this architecture is used as a way to relate our semantics
with R reference interpreter,
it also provided immediate benefits during the development of the semantics.
Section~\ref{sec:driving:development} presents these benefits.
The testing results are shown in Section~\ref{sec:test:results}.
Finally, Section~\ref{sec:proofs} presents some proofs that have been done
using our language formalization.

\section{Coq Interpreter}
\label{sec:coq:interp}

\subsection{Eyeball Closeness}
\label{sec:eyeball:closeness}

We extensively used state + error monads as well as monadic notations
to make every one or two lines of Coq corresponds to one or two lines of C.

Example of C definition / Coq definition.

Size of the project.

\subsection{Structure of the Interpreter}
\label{sec:coq:structure}

C and Coq are widely different programming languages.

A simple model for C's heap, with unions.
Accesses are unguarded in C, but guarded in Coq.

The [runs] trick to support looping in Coq in a structured way (similar to JSRef).

\subsection{Shim}
\label{sec:shim}

Issues with parsing.
How the parser itself is in a one-to-one correspondance with the original R parser.
Why this still doesn't prevent us from difference in behavior.

\section{Testing Architecture}
\label{sec:testing:architecture}

Two goals.
First providing trust to the formalization by certifying the absence of bugs.
Second, help the development of the formalization by catching bugs early.

\subsection{Methodology}
\label{sec:test:methodology}

Various kinds of tests (multiline, line-by-line, tests that are expected to fail, etc.).

What is considered to be a failure.

Including the base library.

\subsection{Driving the Development Process}
\label{sec:driving:development}

Identifying low-hanging fruits.

How the other results (Potential fail, Not implemented) helped the Coq development.

\subsection{Results}
\label{sec:test:results}

How much tests are passed and failed.
What does this mean (one line can trigger more than one pass).

Bugs found (and where).

We extended the testing framework during development by adding new kinds of recognised
data structure.
We consider that the amount of work to extend the framework in another direction
(thus reducing the number of Unknown) is sufficiently low.

\section{Proofs}
\label{sec:proofs}

Tactic development.
Given the size of the formalization, some of these proofs would never have been possible
without some proof automation.
Automation metric: size of the .vo / size of the .v.

Examples of properties that we have proven with our formalization.

Example of tactic in action:
computeR after an allocation updating all the \mintinline{Coq}{safe_pointer}, for instance.

\section{Related Work}
\label{sec:related:work}

R is a notably difficult programming language~\parencite{RInferno},
whose semantics is constantly moving—%
see for instance the recent addition
of R's alternative representation~\parencite{altrepR}.
In our formalization, we chose to ignore these fast moving parts,
but these parts are used by real-world R programs.
Furthermore, the diversity of R users is such that different R programs
will use very different libraries and features,
as the generally accepted guideline~\parencite{RGuidelines}
doesn't restrain users about them.
This makes tools difficult to build for R,
and as a consequence, relatively few tools for R exist
in comparison to the size of its community.

As a consequence,
there exist few testing framework in R.
The testR~\parencite{maj2013testr, 2014testr} project,
which later evolved into Genthat~\parencite{genthat} library,
is however based on an interesting way of generating unit tests for R functions.
It start from a program using the functions to be tested.
It then annotates and executes this program,
storing the trace of the calls to the functions to be tested.
Unit tests are then generated from this trace.
The Genthat library is thus useful to generate tests for a library
given some program using it,
which can then be used to ensure that further versions of the library
don't break already existing code.

GNU~R is not the only R interpreter that exist.
Many existing interpreters are based on the same C core code,
but use different libraries for linear algebra,
usually optimised for a given usage.
%
The FastR~\parencite{kalibera2014fast} project takes a different approach
as it also reimplemented the core code.
It is based on the Truffle~\parencite{wuerthingertruffle}
self-optimizing framework.
Their tool is faster than the reference interpreter
in the language interpretation, and not only the linear-algebraic part.
%
In both cases, a difference of behavior between GNU~R
and the specialized interpreter is considered as a bug.
We believe that this support our choice of strongly basing
our work on this interpreter.

To the extent of our knowledge,
we are the first to provide a mechanized specification of R.
But the general goal of formalizing full real-world languages—%
as opposed to small subsets—is not new.
%
JavaScript is a particular relevant example.
Indeed, empirical analyses~\parencite{RichardsHBV11}
have confirmed that the language features
that are usually ignored in the formalised subsets of R
are actually important for actual web developers.
%
In the case of JavaScript, there are several trust sources.
First, the language is precisely specified by the ECMAScript specification~\parencite{es2019}.
Second, there exist various test suites~\parencite{test262, mozillatests}
as well as several widely used interpreters.
As a consequence, various formal specifications of JavaScript exist,
each related with different of its trust sources.

The first full formal semantics~\parencite{aplas08}
is a semantics related to the third version of the ECMAScript specification.
It had a major influence on the definitions of further JavaScript formal
specification~\parencite{ses, jscert, popl12-Towards, usenix}.
It served as the formal basis to prove the soundness of security-related
JavaScript subsets~\parencite{MMT-CSF-TR09, mmt-esorics09, mmt-oakland10}.
This work was however not mechanized, making it difficult to be used
as a basis for other formal works.

In parallel, several formal semantics~\parencite{js-ml, Guha2010, Politz:S5, kjs}
for JavaScript were based on a JavaScript interpreter.
These semantics are related to JavaScript test suites,
either by comparing the results with the expected result,
or by comparing results with widely used JavaScript interpreters.
These formalizations tend to be easier to build
as testing frameworks already exist.
Furthermore, they are usually easier to understand by non-specialists.
However, such formalizations suffer from all the issues of test suites:
in JavaScript, the \mintinline{javascript}{for}-\mintinline{javascript}{in}
feature was then loosely tested,
and its behavior varied from interpreters to interpreters.

The JSCert formalization~\parencite{jscert, popl14jscert}
is an interesting step forward as it was designed
to be related with both the ECMAScript specification and the JavaScript test suites.
The formalization is composed of two parts:
a mechanized specification and an interpreter.
The JSCert specification is syntactically related with the ECMAScript specification
and the interpreter passes its test suites.
The specification and the interpreter are related to each other by a Coq proof.
This double-relation provides a large amount of trust to JSCert.
In practice, both relations served to find issues in the JSCert specification,
but also some implementation bugs in other interpreters,
as well as mistakes in the ECMAScript specification.
Furthermore, JSCert is mechanized:
this facilitates its reuse for other projects.
However, this project involved 8 people for a year:
building both a specification and an interpreter,
as well as a correctness proof between them, involves a lot of resources.
We solved this issue in the CoqR specification
by defining a denotational semantics:
the same definition is both executable and syntactically close to its specification.

The Coq proof assistant has already been used
in a variety of mechanized language formalization projects.
The most known is the CompCert project~\cite{Leroy-Compcert-CACM}:
this project features an optimizing compiler for C.
This compiler is proven to be free of compilation bugs,
leading to safer programs in critical software.
This projects comes with a formalization of the C programming language,
as well as the formalizations of the intermediate compilation languages.
%
Due to the compiling nature of the CompCert project,
it was acceptable to restrict the behaviors of the C programming language
in their formalization,
restricting it to the behaviors that will actually be compiled by CompCert.
The Formalin project~\parencite{formalin} is another formalization
of the C language:
it aims at precisely listing all the possible behaviors of a C program.

\section{Conclusion and Future Work}
\label{sec:conclusion}

We have a fully trustable formalization of R.

This formalization can be used to prove program logic in R,
the amount of work for a direct approach being quite large.

We have a testing architecture that can be extended.

We believe our testing framework to be adaptable to other situations,
typically another programming language to be tested.

Link with Formalin

\printbibliography{}

\end{document}

