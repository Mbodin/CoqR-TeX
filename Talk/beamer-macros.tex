
\usepackage{tikz}

\usetikzlibrary{trees, arrows, shapes, automata, petri}
\usetikzlibrary{fit, calc, decorations}
\usetikzlibrary{decorations.pathreplacing}
\usetikzlibrary{decorations.pathmorphing}
\usetikzlibrary{positioning}
\usetikzlibrary{shadings, fadings, patterns}
%\usepackage{pgfbaselayers}
\pgfdeclarelayer{background}
\pgfdeclarelayer{foreground}
\pgfsetlayers{background,main,foreground}

% COLORS (Tango)
\definecolor{LightButter}{rgb}{0.98,0.91,0.31}
\definecolor{LightOrange}{rgb}{0.98,0.68,0.24}
\definecolor{LightChocolate}{rgb}{0.91,0.72,0.43}
\definecolor{LightChameleon}{rgb}{0.54,0.88,0.20}
\definecolor{LightSkyBlue}{rgb}{0.45,0.62,0.81}
\definecolor{LightPlum}{rgb}{0.68,0.50,0.66}
\definecolor{LightScarletRed}{rgb}{0.93,0.16,0.16}
\definecolor{Butter}{rgb}{0.93,0.86,0.25}
\definecolor{Orange}{rgb}{0.96,0.47,0.00}
\definecolor{Chocolate}{rgb}{0.75,0.49,0.07}
\definecolor{Chameleon}{rgb}{0.45,0.82,0.09}
\definecolor{SkyBlue}{rgb}{0.20,0.39,0.64}
\definecolor{Plum}{rgb}{0.46,0.31,0.48}
\definecolor{ScarletRed}{rgb}{0.80,0.00,0.00}
\definecolor{DarkButter}{rgb}{0.77,0.62,0.00}
\definecolor{DarkOrange}{rgb}{0.80,0.36,0.00}
\definecolor{DarkChocolate}{rgb}{0.56,0.35,0.01}
\definecolor{DarkChameleon}{rgb}{0.30,0.60,0.02}
\definecolor{DarkSkyBlue}{rgb}{0.12,0.29,0.53}
\definecolor{DarkPlum}{rgb}{0.36,0.21,0.40}
\definecolor{DarkScarletRed}{rgb}{0.64,0.00,0.00}
\definecolor{Aluminium1}{rgb}{0.93,0.93,0.92}
\definecolor{Aluminium2}{rgb}{0.82,0.84,0.81}
\definecolor{Aluminium3}{rgb}{0.73,0.74,0.71}
\definecolor{Aluminium4}{rgb}{0.53,0.54,0.52}
\definecolor{Aluminium5}{rgb}{0.33,0.34,0.32}
\definecolor{Aluminium6}{rgb}{0.18,0.20,0.21}

\setbeamercolor{alerted text}{fg=DarkScarletRed}
\setbeamercolor{background canvas}{bg=white}
\setbeamercolor{block body alerted}{bg=normal text.bg!90!black}
\setbeamercolor{block body}{bg=normal text.bg!90!black}
\setbeamercolor{block body example}{bg=normal text.bg!90!black}
\setbeamercolor{block title alerted}{use={normal text,alerted text},fg=alerted text.fg!75!normal text.fg,bg=normal text.bg!75!black}
\setbeamercolor{block title}{bg=Plum,fg=Aluminium1}
\setbeamercolor{block title example}{use={normal text,example text},fg=example text.fg!75!normal text.fg,bg=normal text.bg!75!black}
\setbeamercolor{fine separation line}{}
\setbeamercolor{frametitle}{fg=Aluminium1,bg=LightPlum}
\setbeamercolor{item projected}{fg=Aluminium1,parent=palette primary}
\setbeamercolor{normal text}{bg=white,fg=Aluminium6}
\setbeamercolor{palette sidebar primary}{use=normal text,fg=normal text.fg}
\setbeamercolor{palette sidebar quaternary}{use=structure,fg=structure.fg}
\setbeamercolor{palette sidebar secondary}{use=structure,fg=structure.fg}
\setbeamercolor{palette sidebar tertiary}{use=normal text,fg=normal text.fg}
\setbeamercolor{section in sidebar}{fg=Chameleon}
\setbeamercolor{section in sidebar shaded}{fg=Chameleon}
\setbeamercolor{separation line}{}
\setbeamercolor{sidebar}{bg=Chameleon}
\setbeamercolor{sidebar}{parent=palette primary}
\setbeamercolor{structure}{bg=Aluminium1,fg=Plum}
\setbeamercolor{subsection in sidebar}{fg=Chameleon}
\setbeamercolor{subsection in sidebar shaded}{fg=Chameleon}
\setbeamercolor{title}{fg=Aluminium1,bg=LightPlum}
\setbeamercolor{titlelike}{fg=Aluminium1}


\tikzset{
	locnode base/.style = {
		rectangle,
		rounded corners = 2pt,
		text centered,
		minimum height = 5mm,
		minimum width = 10mm,
        bottom color = Aluminium1
	},
    locnode grey/.style = {
        locnode base,
		draw = Aluminium6,
		top color = Aluminium1,
		bottom color = Aluminium2
    },
    locnode/.style = {
        locnode grey
    },
	locnode orange/.style = {
        locnode base,
        draw = DarkOrange,
		top color = LightOrange,
		bottom color = Orange
	},
	locnode green/.style = {
        locnode base,
        draw = DarkChameleon,
		top color = Aluminium1,
		bottom color = Chameleon
	},
	locnode red/.style = {
        locnode base,
        draw = DarkScarletRed,
		top color = Aluminium1,
		bottom color = LightScarletRed
	},
	locnode plum/.style = {
        locnode base,
        draw = DarkPlum,
		top color = Aluminium1,
		bottom color = LightPlum
	},
	locnode blue/.style = {
        locnode base,
        draw = DarkSkyBlue,
		top color = Aluminium1,
		bottom color = LightSkyBlue
	},
	locnode brown/.style = {
        locnode base,
        draw = DarkChocolate,
		top color = Aluminium1,
		bottom color = LightChocolate
	},
	locnode yellow/.style = {
        locnode base,
        draw = DarkButter,
		top color = Aluminium1, % LightButter,
		bottom color = Butter
	}
}


\usepackage{beamerthemesplit}
\usepackage{wasysym}

\usetheme{Copenhagen}
%\usecolortheme{crane}
%\useoutertheme{infolines}
% COLORS (Tango)
\definecolor{LightButter}{rgb}{0.98,0.91,0.31}
\definecolor{LightOrange}{rgb}{0.98,0.68,0.24}
\definecolor{LightChocolate}{rgb}{0.91,0.72,0.43}
\definecolor{LightChameleon}{rgb}{0.54,0.88,0.20}
\definecolor{LightSkyBlue}{rgb}{0.45,0.62,0.81}
\definecolor{LightPlum}{rgb}{0.68,0.50,0.66}
\definecolor{LightScarletRed}{rgb}{0.93,0.16,0.16}
\definecolor{Butter}{rgb}{0.93,0.86,0.25}
\definecolor{Orange}{rgb}{0.96,0.47,0.00}
\definecolor{Chocolate}{rgb}{0.75,0.49,0.07}
\definecolor{Chameleon}{rgb}{0.45,0.82,0.09}
\definecolor{SkyBlue}{rgb}{0.20,0.39,0.64}
\definecolor{Plum}{rgb}{0.46,0.31,0.48}
\definecolor{ScarletRed}{rgb}{0.80,0.00,0.00}
\definecolor{DarkButter}{rgb}{0.77,0.62,0.00}
\definecolor{DarkOrange}{rgb}{0.80,0.36,0.00}
\definecolor{DarkChocolate}{rgb}{0.56,0.35,0.01}
\definecolor{DarkChameleon}{rgb}{0.30,0.60,0.02}
\definecolor{DarkSkyBlue}{rgb}{0.12,0.29,0.53}
\definecolor{DarkPlum}{rgb}{0.36,0.21,0.40}
\definecolor{DarkScarletRed}{rgb}{0.64,0.00,0.00}
\definecolor{Aluminium1}{rgb}{0.93,0.93,0.92}
\definecolor{Aluminium2}{rgb}{0.82,0.84,0.81}
\definecolor{Aluminium3}{rgb}{0.73,0.74,0.71}
\definecolor{Aluminium4}{rgb}{0.53,0.54,0.52}
\definecolor{Aluminium5}{rgb}{0.33,0.34,0.32}
\definecolor{Aluminium6}{rgb}{0.18,0.20,0.21}

\setbeamercolor{alerted text}{fg=DarkScarletRed}
\setbeamercolor{background canvas}{bg=white}
\setbeamercolor{block body alerted}{bg=normal text.bg!90!black}
\setbeamercolor{block body}{bg=normal text.bg!90!black}
\setbeamercolor{block body example}{bg=normal text.bg!90!black}
\setbeamercolor{block title alerted}{use={normal text,alerted text},fg=alerted text.fg!75!normal text.fg,bg=normal text.bg!75!black}
\setbeamercolor{block title}{bg=Plum,fg=Aluminium1}
\setbeamercolor{block title example}{use={normal text,example text},fg=example text.fg!75!normal text.fg,bg=normal text.bg!75!black}
\setbeamercolor{fine separation line}{}
\setbeamercolor{frametitle}{fg=Aluminium1,bg=LightPlum}
\setbeamercolor{item projected}{fg=Aluminium1,parent=palette primary}
\setbeamercolor{normal text}{bg=white,fg=Aluminium6}
\setbeamercolor{palette sidebar primary}{use=normal text,fg=normal text.fg}
\setbeamercolor{palette sidebar quaternary}{use=structure,fg=structure.fg}
\setbeamercolor{palette sidebar secondary}{use=structure,fg=structure.fg}
\setbeamercolor{palette sidebar tertiary}{use=normal text,fg=normal text.fg}
\setbeamercolor{section in sidebar}{fg=Chameleon}
\setbeamercolor{section in sidebar shaded}{fg=Chameleon}
\setbeamercolor{separation line}{}
\setbeamercolor{sidebar}{bg=Chameleon}
\setbeamercolor{sidebar}{parent=palette primary}
\setbeamercolor{structure}{bg=Aluminium1,fg=Plum}
\setbeamercolor{subsection in sidebar}{fg=Chameleon}
\setbeamercolor{subsection in sidebar shaded}{fg=Chameleon}
\setbeamercolor{title}{fg=Aluminium1,bg=LightPlum}
\setbeamercolor{titlelike}{fg=Aluminium1}


\tikzset{
	locnode base/.style = {
		rectangle,
		rounded corners = 2pt,
		text centered,
		minimum height = 5mm,
		minimum width = 10mm,
        bottom color = Aluminium1
	},
    locnode grey/.style = {
        locnode base,
		draw = Aluminium6,
		top color = Aluminium1,
		bottom color = Aluminium2
    },
    locnode/.style = {
        locnode grey
    },
	locnode orange/.style = {
        locnode base,
        draw = DarkOrange,
		top color = LightOrange,
		bottom color = Orange
	},
	locnode green/.style = {
        locnode base,
        draw = DarkChameleon,
		top color = Aluminium1,
		bottom color = Chameleon
	},
	locnode red/.style = {
        locnode base,
        draw = DarkScarletRed,
		top color = Aluminium1,
		bottom color = LightScarletRed
	},
	locnode plum/.style = {
        locnode base,
        draw = DarkPlum,
		top color = Aluminium1,
		bottom color = LightPlum
	},
	locnode blue/.style = {
        locnode base,
        draw = DarkSkyBlue,
		top color = Aluminium1,
		bottom color = LightSkyBlue
	},
	locnode brown/.style = {
        locnode base,
        draw = DarkChocolate,
		top color = Aluminium1,
		bottom color = LightChocolate
	},
	locnode yellow/.style = {
        locnode base,
        draw = DarkButter,
		top color = Aluminium1, % LightButter,
		bottom color = Butter
	}
}


\setbeamertemplate{headline}{}
\setbeamertemplate{footline}{}

\setbeamertemplate{navigation symbols}{%
    %\insertdocnavigationsymbol
    \ifnum\theframenumber=0\relax\else%
        \normalsize\textcolor{DarkPlum}{\insertframenumber}%/\inserttotalframenumber}
    \fi%
    %\quad
}

\makeatletter
%\sectionframe[]{}
% no star: section name (facultative, if not specified, it is the second argument), the huge text written on the slide.
% one star: no section defined, the argument is the huge text.
% two stars: no section defined, the arguments are the huge text and some text to be written afterwards.
% I guess {\sectionframe}*: a if no stars, but with an additional argument, with is the text written afterwards.
\newcommand\sectionframe@star@add[2]{
    \begin{frame}
        \begin{center}
            \Huge \textcolor{Plum}{#1}
        \end{center}
        #2
    \end{frame}
}
\newcommand\sectionframe@unstar@add[3][]{
    \ifempty{#1}{\section{#2}}{\section{#1}}
    \sectionframe@star@add{#2}{#3}
}
\newcommand\sectionframe@star@base[1]{\sectionframe@star@add{#1}{}}
\newcommand\sectionframe@unstar@base[2][]{\sectionframe@unstar@add[#1]{#2}{}}
\newcommand\sectionframe@star{\@ifstar\sectionframe@star@add\sectionframe@star@base}
\newcommand\sectionframe@unstar{\@ifstar\sectionframe@unstar@add\sectionframe@unstar@base}
\newcommand\sectionframe{\@ifstar\sectionframe@star\sectionframe@unstar}
\makeatother

\newcommand<>\Alt[2]{{%
        \sbox0{$\displaystyle #1$}%
        \sbox1{$\displaystyle #2$}%
        \alt#3%
        {\rlap{\usebox0}\vphantom{\usebox1}\hphantom{\ifnum\wd0>\wd1 \usebox0\else\usebox1\fi}}%
        {\rlap{\usebox1}\vphantom{\usebox0}\hphantom{\ifnum\wd0>\wd1 \usebox0\else\usebox1\fi}}%
}}

\makeatletter
\newcommand\bsphere[2][-.65ex]{%
    \raisebox{#1}{%
    \begin{pgfpicture}[-1ex]{-.65ex}{1ex}{1ex}%
       \usebeamercolor[fg]{item projected}%
       {\pgftransformscale{1.75}\pgftext{\normalsize\pgfuseshading{bigsphere}}}%
       {\pgftransformshift{\pgfpoint{0pt}{.5pt}}%
           \pgftext{\usebeamerfont*{item projected}%
               \usebeamercolor[fg]{item projected}%
               #2%
           }%
       }%
    \end{pgfpicture}%
   }%
}
\makeatother

\newcommand\provedincoq[1]{%
    \vspace{-6.5mm}%
    \hfill%
    \parbox[b][0pt][b]{2cm}{%
        \hfill%
        \ifempty{#1}{%
            \includegraphics[height = 8mm]{images/logo_tampon_3d.png}%
        }{%
            ~\vspace{-2mm}%
            \anchor{#1}{%
                \includegraphics[height = 8mm]{images/logo_tampon_3d.png}%
            }%
            \hspace{-4mm}%
        }%
        \hspace{1mm}~%
    }%
    \vspace{4mm}%
}

\newenvironment{variableblock}[3]{%
    \setbeamercolor{block body}{#2}
    \setbeamercolor{block title}{#3}
          \begin{block}{#1}}{\end{block}}

\newenvironment{innertikz}[1][]{%
    \begin{tikzpicture}[baseline={([yshift=-.5ex]current bounding box.center)}, #1]
}{
    \end{tikzpicture}%
}

\newenvironment{centertikz}[1][]{%
    \begin{center}%
    \begin{tikzpicture}[#1]
}{
    \end{tikzpicture}%
    \end{center}
}

\def\changemargin#1#2{\list{}{\rightmargin#2\leftmargin#1}\item[]}
\let\endchangemargin=\endlist 

\newenvironment{widemargin}{
    \begin{changemargin}{-1cm}{-1cm}}{
    \end{changemargin}}

\newcommand\anchor[3][]{%
    \begin{innertikz}[remember picture, #1]%
        \node (#2) {#3} ;%
    \end{innertikz}%
}

\newcommand*\anchorm[3][]{%
    \anchor[#1]{#2}{\!\!\(#3\)\!\!}%
}

