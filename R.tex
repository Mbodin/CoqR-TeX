\documentclass{article}

\usepackage[
   backend=biber,
   style=alphabetic,
   citestyle=authoryear-comp,
   sortlocale=en,
   url=true, 
   doi=true,
   eprint=false
 ]{biblatex}
\addbibresource{biblio.bib}

\newcommand\R{R}

\title{Notes about \R}
\author{Martin Bodin}

\begin{document}

\maketitle

\section{Presentation of the Language}

\subsection{History}

\R{} is a programming language designed for statistics.
Its specification~\parencite{team2000r} is not precise:
in practise, it is specified by its main implementation~\parencite{Rwebsite}.

The original authors of \R{}~\parencite{ihaka1996r}
describe \R{} as a programming language similar to Scheme
which has been mutated to get a programming language similar to S.
In particular, \R{} features scoping and first-class functions
as Scheme.
Similarly to S, however, it is a lazy programming language.

It is a community driven programming language.
This means that most of what is used in a \R{} program comes from various libraries,
which can change the way the programming language behaves.
This can be compared with JavaScript,
and how libraries like jQuery changes the way programs look.

\subsection{Features}

\subsubsection{Lazyness}


\subsubsection{Array Filtering}


\section{\R{} Interpreters}

\subsection{The Main \R{} Interpreter}

\subsubsection{Concepts}

Most \R{} objects are in the form of a \emph{basic language element}.
This is a C structure composed of a tag and four pointers.
The tag precises the kind of the basic language element;
it can be for instance an integer vector, an environment, an expression, or an external pointer.
The meaning of the last three vectors depends on the flag.
%
For instance, for an environment, the first pointer is a pointer
to the current frame (associating each local variable to a value or a promise),
the second pointer points to the environment of the outer scope,
and the third pointer points to a hash (to enable faster checks).
%
For a list, the first pointer points to the first element of the list,
the second, to the rest of the list,
and the third, to an optional name for the first element.

\subsubsection{Main Files and Functions}


\subsection{FastR}

The \R{} runtime can be slow.
\cite{kalibera2014fast} propose a faster approach.


\printbibliography

\end{document}

