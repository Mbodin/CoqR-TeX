


\begin{frame}
    \frametitle{JSCert Overview}

\end{frame}

\sectionframe{R}

\begin{frame}
    \frametitle{R: A Programming Language About Vectors}

% a[1], a[index], a[c(false, true, false)]…

\end{frame}

\begin{frame}
    \frametitle{R: A Lazy Programming Language}

\end{frame}

\begin{frame}
    \frametitle{R: A Dynamic Programming Language}

% Changing the meaning of parentheses.
% The keyword “function” is… a function (working thanks to lazyness!).

\end{frame}

\begin{frame}
    \frametitle{R Trust Sources}

% No specification (yet).
% A reference interpreter.

\end{frame}

\begin{frame}
    \frametitle{Inside GNU R}

% C.
% Separation between core and non-core.

\end{frame}

\begin{frame}
    \frametitle{Formalising R Core}

% Monad state + error.

\end{frame}

\begin{frame}
    \frametitle{Snippets of GNU R}

% SETJMP

\end{frame}

\begin{frame}
    \frametitle{Future}

% Testing.
% Link using CompCert/Formalin.
% High-level specification.
    % To be useful for Coq users (show the proof of 1+1→2 in JavaScript)
    % To be useful for the R community (RExplain?).

\end{frame}

